Setting up Transport Layer Security (TLS) is an advanced topic which requires
owning a domain, having control over a DNS server, the ability to forward ports
on the edge router, setting up web servers, and using the command line.  As
such, it is recommended that only people who are at least somewhat familiar
with these technologies attempt to set this up.  For the vast majority of
users, setting up TLS is not necessary, and it is safe to skip this step.

At the core, setting up TLS just means giving the server a host/domain name,
getting a trusted certificate for that name, and accessing the server by name
instead of by IP address.

TODO: Draw a diagram showing what's going on so it's clear how everything fits
together.

TODO: Document setting a static IP address on the HestiaPi

TODO: Document setting up DNS using Bind9

TODO: Document obtaining a TLS certificate from LetsEncrypt

TODO: Document configuring the HestiaPi to use the newly obtained certificate

