On startup, the HestiaPi will run a number of services, which does take some
time due to the low computational power of the pi.  However, booting is
something which very infrequently done, which is why the boot times of five
minutes or more are acceptable in order to keep the size and cost of the
hardware low.

Understanding the startup process requires understanding what components are
running.  The components are documented in ???.

Systemd starts up a number of services, including:
\begin{enumerate}
  \item mosquitto
  \item hciuart
  \item triggerhappy
  \item dhcpcd
  \item openhab2
\end{enumerate}

The status of sll of these services can be checked with
\texttt{systemctl status SERVICENAME} where SERVICENAME is replaced with the
name of the service of interest. For example, to check the status of Mosquitto:
\texttt{systemctl status mosquitto}

In addition to things started by systemd, there are also scripts which are run
from /etc/rc.local.

