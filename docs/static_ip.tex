Setting a static IP address on a Raspberry Pi can be done with either the
dhcpcd or the networking service.  These are documented in more detail
\href{https://raspberrypi.stackexchange.com/questions/37920/how-do-i-set-up-networking-wifi-static-ip-address/74428#74428}
{on this stackexchange post}, so this guide only goes over the basics.  The
dhcpcd option has been thoroughly tested with the HestiPi in particular, but
the networking option should also work, as that has been tested with Raspberry
Pis in general.

A third option is to get a dynamic IP address every time the HestiaPi starts
up, but configure your router to look at the MAC address that the request came
from and always assign the same dynamic address.  This process will be
different for every router and can not be covered in this documentation.  For
instructions on how to do this, consult your router's documentation.

\subsubsection{Configuring dhcpcd}
Using dhcpcd to configure a static IP address requires editing
\texttt{/etc/dhcpcd.conf} and restarting the dhcpcd service.  The configuration
file contains some commented out examples, which look similar to this:

\begin{verbatim}
interface wlan0
static ip_address=10.1.1.31/24
static routers=10.1.1.1
static domain_name_servers=10.1.1.1
\end{verbatim}

Then run \texttt{sudo systemctl restart dhcpcd} to start using the new
configuration.  The system will then have two IP addresses, the DHCP assigned
one that it already had, and the newly assigned one.  This is because dhcpcd
doesn't remove any IP addresses.  When the HestiaPi is rebooted, it will only
have the static IP address.

\subsubsection{Configuring Networking Service}
The other option is to just disable DHCP all together and set the static IP
address in \texttt{/etc/network/interfaces}.  The configuration would look
similar to the example below:

\begin{verbatim}
iface wlan0 inet static
    address 10.1.1.31
    netmask 255.255.255.0
    gateway 10.1.1.1
    wpa-conf /etc/wpa_supplicant/wpa_supplicant.conf
\end{verbatim}

Once this is set, run \texttt{sudo systemctl disable dhcpcd} to make sure
it doesn't cause any issues, then run \texttt{sudo systemctl enable networking}
This should reconfigure the interface to use only the staticly assigned IP
address.  Unlike the dhcpcd option, this means your SSH connection will be
terminated and you will have to SSH into the machine again, but this also
avoids a reboot (however rebooting is still recommended to verify that the
IP address is set properly on boot).

