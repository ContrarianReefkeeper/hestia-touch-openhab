\textbf{WiFi details}

\texttt{/etc/wpa\_supplicant/wpa\_supplicant.conf}

\textbf{OpenHAB}
Items

\texttt{/etc/openhab2/items/default.items}

Rules

\texttt{/etc/openhab2/rules/default.rules}

Sitemaps

\texttt{/etc/openhab2/sitemaps/default.sitemap}

Things

\texttt{/etc/openhab2/things/default.things}

Logs

\texttt{/var/log/openhab2/events.log\\
/var/log/openhab2/openhab.log}

\textbf{LCD UI}
The LCD UI is an HTML-based page loaded on a fullscreen browser. All HTML, CSS, JS, fonts and icon files are in here

\texttt{/home/pi/scripts/oneui}

The vue framework is used.
    
\textbf{Scripts}
In 
\texttt{/home/pi/scripts}

There are
\texttt{AdafruitDHTHum.py\\
AdafruitDHTTemp.py}

Read sensor data from DHT sensors.

\texttt{C2F.sh\\
F2C.sh}

Change HestiaPi from Celcius to Fahrenheit and vice versa.

\texttt{getBMEhumi.sh\\
getBMEtemp.sh\\
getBMEpress.sh}

Read sensor data from BME sensors (calling bme280.py).

\texttt{getcputemperature.sh}

Returns RasPi CPU temperature.

\texttt{getssid.sh}

Returns WiFi SSID name.

\texttt{gettz.sh}

Returns system Timezone.

\texttt{getuseddiskspace.sh}

Returns used SD card space.

\texttt{getwifiinfo.sh }

Returns WiFi signal strength.

\texttt{getwlan0ip.sh }

Returns WiFi IP.

\texttt{getwlan0mac.sh }

Returns WiFi MAC address.

\texttt{netcheck.sh }

Cron script that checks WiFi connectivity by pinging its gateway. If no
response is received at the first time, the WiFi interface is restarted and a
DHCP (dynamic) IP is requested. If no response is received again RaspberryPi,
the reboot command is sent. Please note this script is not enabled by default
and you will need to follow the instructions supplied at the top of the file.
Please also note that restarting the Pi will stop any current task and will not
resume after restart.

\texttt{openhabloader.sh }

Loads the Touch LCD UI.

\texttt{getpublicip.sh }

Checks current public IP and if it matches with previous reading, it does
nothing else. If current public IP is different, the latest value is sent to
your account (manual and free account registration needed).
    
\textbf{Web UI}

\texttt{http://[YOUR\_HESTIA\_IP]:8080/basicui/app}

or simply


\texttt{http://[YOUR\_HESTIA\_IP]:8080}

and then select Basic UI and default

\textbf{Smartphone App}
Under Settings > Local server settings


\texttt{http://[YOUR\_HESTIA\_IP]:8080}
